% !TEX encoding = UTF-8 Unicode

%%%%%%%%%%%%%%%%%%%%%%%%%%%%%%%%%%%%%%%%%%%%%%%%%%%%%%%%%%%%%%%%%%%%%%%%%%%%%%%%
%%%                         ИНФОРМАЦИЯ ДЛЯ ДОГОВОРА                          %%%
%%%%%%%%%%%%%%%%%%%%%%%%%%%%%%%%%%%%%%%%%%%%%%%%%%%%%%%%%%%%%%%%%%%%%%%%%%%%%%%%

% Дата заключения договора.
\def\date{%
    <<7>> февраля 2023~г.
}

% ФИО декана в родительном падеже,
% номер и дата доверенности (выдается раз в год).
\def\dean{
    Шафаревича Андрея Игоревича,
    действующего на основании
    доверенности №322-22/010-50 от 6 декабря 2022 года
}
\def\deanShort{
    Шафаревич~А.~И.
}

% Наименование организации, учреждения, предприятия.
\def\organizationName{
    ООО <<Органзиация>>
}

% ФИО руководителя организации в родительном падеже,
% наименование основания.
\def\organizationDirector{
    Иванова Ивана Ивановича,
    действующего на основании
    Устава
}
\def\organizationDirectorShort{
    Иванов~И.\,И.
}

% Адрес организации, ИНН, КПП.
\def\organizationAddress{
    123456, г.~Москва, ул.~Ленина, д.~1
}
\def\organizationINN{
    1234567890
}
\def\organizationKPP{
    123456789
}

% Контактное лицо: ФИО, e-mail
% (вместо _ необходимо писать \_), телефон.
\def\organizationContactName{
    Иванов Иван Иванович
}
\def\organizationContactEmail{
    ivanov\_ii@mail.com
}
\def\organizationContactPhone{
    +7~(912)~345-67-89
}

% Список студентов-практикантов.
% Для добавления студентов необходимо
% использовать макрос \addStudet.
\def\studentsList{
    \addStudent{Иванов}{Иван}{Иванович}
    \addStudent{Иванов}{Иван}{Иванович}
    \addStudent{Иванов}{Иван}{Иванович}
}

% Список помещений, оборудования и технических средств.
% Для добавления помещений необходимо использовать макрос \addRoom.
% Можно оставить пустым.
% Для принудительного разрыва строк необходимо использовать \newline
\def\roomsList{
    \addRoom
        {Помещение и оборудование \newline ООО <<Организация>>}
        {г.~Москва, ул.~Ленина, д.~1}
    \addRoom
        {Помещение и оборудование \newline ООО <<Организация>>}
        {г.~Москва, ул.~Ленина, д.~1}
    \addRoom
        {Помещение и оборудование \newline ООО <<Организация>>}
        {г.~Москва, ул.~Ленина, д.~1}
}

% Даты начала и конца практики.
\def\practiceBeginDate{
    7 февраля 2023
}
\def\practiceEndDate{
    28 февраля 2023
}

%%%%%%%%%%%%%%%%%%%%%%%%%%%%%%%%%%%%%%%%%%%%%%%%%%%%%%%%%%%%%%%%%%%%%%%%%%%%%%%%
%%%                                ПРЕАМБУЛА                                 %%%
%%%%%%%%%%%%%%%%%%%%%%%%%%%%%%%%%%%%%%%%%%%%%%%%%%%%%%%%%%%%%%%%%%%%%%%%%%%%%%%%

\documentclass[
    a4paper,
    oneside,
    % draft,
]{amsart}

\usepackage{lipsum}

% Шрифт
\usepackage{tempora}

\usepackage[fontsize=12pt]{scrextend}

\usepackage{setspace}
\onehalfspace

\usepackage[T2A]{fontenc}
\usepackage[utf8]{inputenc}
\usepackage[russian]{babel}
\usepackage{cmap}

\usepackage[
    left=30mm,
    right=10mm,
    top=20mm,
    bottom=20mm,
]{geometry}

\usepackage{soulutf8}

\usepackage{enumitem}
\setlist{nolistsep, noitemsep, leftmargin=30pt, topsep=0pt}

\usepackage{tabularx}
\usepackage{ltablex}
\usepackage{booktabs}

\usepackage{makecell}

\usepackage{calc}

% Математический шрифт
\usepackage[bigdelims, vvarbb]{newtxmath}

% disable indent
\setlength{\parindent}{0pt}

%%%%%%%%%%%%%%%%%%%%%%%%%%%%%%%%%%%%%%%%%%%%%%%%%%%%%%%%%%%%%%%%%%%%%%%%%%%%%%%%
%%%                         ВСПОМОГАТЕЛЬНЫЕ МАКРОСЫ                          %%%
%%%%%%%%%%%%%%%%%%%%%%%%%%%%%%%%%%%%%%%%%%%%%%%%%%%%%%%%%%%%%%%%%%%%%%%%%%%%%%%%

\def\trim#1{%
    \ignorespaces#1\unskip%
}

\def\SECTION#1{%
    % \section*{\normalfont \textbf{#1}}%
    \specialsection*{\textbf{#1}}%
}

\def\tline#1#2{%
    $\underset{\text{#1}}{\text{\underline{\hskip #2}}}$%
}

\def\sign#1{%
    \tline{(подпись)}{100pt}
    {\hyphenchar\font=-1 /~\trim{#1}~/}%
}

\def\dateRangeCell#1#2{%
    \makecell[lt]{#1\,-- \\ #2}%
}

%%%%%%%%%%%%%%%%%%%%%%%%%%%%%%%%%%%%%%%%%%%%%%%%%%%%%%%%%%%%%%%%%%%%%%%%%%%%%%%%
%%%                             ТЕКСТ ДОКУМЕНТА                              %%%
%%%%%%%%%%%%%%%%%%%%%%%%%%%%%%%%%%%%%%%%%%%%%%%%%%%%%%%%%%%%%%%%%%%%%%%%%%%%%%%%

\begin{document}
\thispagestyle{empty}
\pagestyle{empty}

\begin{center}
    \bf
    ДОГОВОР

    \vskip .5\baselineskip

    на прохождение практики студентами Московского государственного университета \\
    имени М.\,В.~Ломоносова в организации (учреждении, предприятии)
\end{center}

\vskip .5\baselineskip

г. Москва \hfill \trim{\date}

\vskip .5\baselineskip

Федеральное государственное бюджетное образовательное учреждение высшего
образования <<Московский государственный университет имени
М.\,В.~Ломоносова>> в лице декана механико\,--\,математического факультета
\trim{\dean}, (далее --- Организация), с одной стороны и
\trim{\organizationName}, (далее --- Профильная организация), в лице
\trim{\organizationDirector} с другой стороны, именуемые по отдельности
<<Сторона>>, а вместе <<Стороны>>, заключили настоящий Договор о
нижеследующем:


\SECTION{1. Предмет договора}

\par 1.1. Предметом настоящего договора является организация практической
    подготовки обучающихся (далее --- практическая подготовка).
\par 1.2. Образовательная программа (программы), компоненты образовательной
    программы, при реализации которых организуется практическая подготовка,
    количество обучающихся, осваивающих соответствующие компоненты
    образовательной программы, сроки организации практической подготовки,
    согласуются Сторонами и являются неотъемлемой частью настоящего договора
    (Приложение №1).
\par 1.3. Реализация компонентов образовательной программы, согласованных
    Сторонами в Приложении №1 к настоящему Договору (далее --- компоненты
    образовательной программы) осуществляется в помещениях Профильной
    организации, перечень которых согласуется Сторонами и является
    неотъемлемой частью настоящего Договора (Приложение №2).


\SECTION{2. Права и обязанности Сторон}

\par 2.1. Организация обязана:
\par 2.1.1. не позднее, чем за 10 дней до начала практической подготовки
    по каждому компоненту образовательной программы предоставить в
    Профильную организацию поименные списки обучающихся, осваивающих
    соответствующие компоненты образовательной программы посредством
    практической подготовки;
\par 2.1.2. назначить руководителя по практической подготовке от
    Организации, который:
    \begin{itemize}
        \item[--] обеспечивает организацию образовательной деятельности в
            форме практической подготовки при реализации компонентов
            образовательной программы;
        \item[--] организует участие обучающихся в выполнении определенных
            видов работ, связанных с будущей профессиональной деятельностью;
        \item[--] оказывает методическую помощь обучающимся при выполнении
            определенных видов работ, связанных с будущей профессиональной
            деятельностью;
        \item[--] несет ответственность совместно с работником Профильной
            организации за реализацию компонентов образовательной программы
            в форме практической подготовки, за жизнь и здоровье обучающихся
            и работников Организации, соблюдении ими правил противопожарной
            безопасности, правил охраны труда, техники безопасности и
            санитарно\,--\,э\-пи\-де\-ми\-о\-ло\-ги\-че\-ских правил и
            гигиенических нормативов;
    \end{itemize}
\par 2.1.3. при смене руководителя по практической подготовке в 14-дневный
    срок сообщить об этом Профильной организации;
\par 2.1.4. установить виды учебной деятельности, практики и иные компоненты
    образовательной программы, осваиваемые обучающимися в форме практической
    подготовки, включая место, продолжительность и период их реализации;
\par 2.1.5. направить обучающихся в Профильную организацию для освоения
    компонентов образовательной программы в форме практической подготовки;
\par 2.1.6. \tline{(иные обязанности Организации)}{350pt}.
\par 2.2. Профильная организация обязана:
\par 2.2.1. создать условия для реализации компонентов образовательной
    программы в форме практической подготовки, предоставить оборудование и
    технические средства в объеме, позволяющие выполнить определенные виды
    работ, связанные с будущей профессиональной деятельностью обучающихся;
\par 2.2.2. назначить ответственное лицо, соответствующее требованиям
    трудового законодательства Российской Федерации о допуске к
    педагогической деятельности, из числа работников Профильной организации
    (Приложение №2 к настоящему Договору), которое обеспечивает организацию
    реализации компонентов образовательной программы в форме практической
    подготовки со стороны Профильной организации;
\par 2.2.3. при смене лица, указанного в пункте 2.2.2, в 14-дневный срок
    сообщить об этом в Организацию;
\par 2.2.4. обеспечить безопасные условия реализации компонентов
    образовательной программы в форме практической подготовки, выполнение
    правил противопожарной безопасности, правил охраны труда, техники
    безопасности и санитарно\,--\,э\-пи\-де\-ми\-о\-ло\-ги\-че\-ских правил
    и гигиенических нормативов;
\par 2.2.5. проводить оценку охраны труда на рабочих местах при реализации
    компонентов образовательной программы в форме практической подготовки,
    и сообщать руководителю Организации об условиях труда и требованиях
    охраны труда на рабочем месте;
\par 2.2.6. ознакомить обучающихся с правилами внутреннего трудового
    распорядка Профильной организации, \tline{(указываются иные локальные
    нормативные акты Профильной организации)}{350pt};
\par 2.2.7. провести инструктаж обучающихся по охране труда и технике
    безопасности и осуществлять надзор за соблюдением обучающимися правил
    техники безопасности;
\par 2.2.8. предоставить обучающимся и руководителю по практической
    подготовке от Организации возможность пользоваться помещениями
    Профильной организации, согласованными Сторонами (Приложение №2 к
    настоящему Договору), а также находящимся в них оборудованием и
    техническими средствами обучения;
\par 2.2.9. обо всех случаях нарушения обучающимися правил внутреннего
    трудового распорядка, охраны труда и техники безопасности сообщить
    руководителю по практической подготовке от Организации;
\par 2.2.10. \tline{(иные обязанности Профильной организации)}{350pt}.
\par 2.3. Организация имеет право:
\par 2.3.1. осуществлять контроль соответствия условий реализации
    компонентов образовательной программы в форме практической подготовки
    требованиям настоящего Договора;
\par 2.3.2. запрашивать информацию об организации практической подготовки, в
    том числе о качестве и объёме выполненных обучающимися работ, связанных
    с будущей профессиональной деятельностью;
\par 2.3.3. \tline{(иные права Организации)}{350pt}.
\par 2.4. Профильная организация имеет право:
\par 2.4.1. требовать от обучающихся соблюдения правил внутреннего трудового
    распорядка, охраны труда и техники безопасности, режима
    конфиденциальности, принятого в Профильной организации, предпринимать
    необходимые действия, направленные на предотвращения ситуаций,
    способствующих разглашению конфиденциальной информации;
\par 2.4.2. в случае установления факта нарушения обучающимися своих
    обязанностей в период организации практической подготовки, режима
    конфиденциальности приостановить реализацию компонентов образовательной
    программы в форме практической подготовки в отношении конкретного
    обучающегося;
\par 2.4.3. \tline{(иные права Профильной организации)}{350pt}.


\SECTION{3. Срок действия договора}

\par 3.1. Настоящий Договор вступает в силу с момента подписания его
    Сторонами и действует в течение 5 (пяти) лет. Если по истечении
    указанного срока ни одна из Сторон не заявит о намерении расторгнуть
    Договор или заключить его на новых условиях, действие Договора может
    быть пролонгировано на тот же срок по соглашению Сторон.
\par 3.2. Настоящий Договор может быть досрочно расторгнут по соглашению
    Сторон или в одностороннем порядке с письменным предупреждением другой
    Стороны о расторжении Договора за 4 (четыре) месяца до предполагаемой
    даты расторжения.


\SECTION{4. Заключительные положения}

\par 4.1. Все споры, возникающие между Сторонами по настоящему Договору,
    разрешаются в установленном порядке согласно действующему
    законодательству Российской Федерации.
\par 4.2. Изменения настоящего Договора осуществляется по соглашению Сторон
    в письменной форме в виде дополнительных соглашений к настоящему
    Договору, которые являются его неотъемлемой частью. Рассматриваются
    изменения в месячный срок с момента получения Стороной соответствующего
    предложения.
\par 4.3. Настоящий Договор составлен в двух экземплярах, по одному для
    каждой из Сторон. Все экземпляры имеют равную юридическую силу.


\SECTION{5. Адреса, реквизиты и подписи Сторон}

% Все содержимое 5 раздела договора дублируется в каждом из двух приложений,
% поэтому текст обернут в макрос.

\def\adresstable{
    \noindent
    \begin{minipage}[t]{.48\textwidth}
        \organizationName \\[.5\baselineskip]
        Юридический адрес: \\
        \trim{\organizationAddress} \\
        ИНН \trim{\organizationINN} / КПП \trim{\organizationKPP} \\[.5\baselineskip]
        Контактное лицо: \\
        ФИО: \trim{\organizationContactName} \\
        E-mail: \texttt{\trim{\organizationContactEmail}} \\
        Тел.: \trim{\organizationContactPhone} \\[.5\baselineskip]
        % \organizationDirectorPosition \\[5pt]
        \textbf{Руководитель профильной организации} \\[5pt]
        \sign{\organizationDirectorShort} \\[40pt]
        м.~п.
    \end{minipage}
    \hfill
    \begin{minipage}[t]{.48\textwidth}
        Федеральное государственное бюджетное образовательное учреждение
        высшего образования <<Московский государственный университет имени
        М.\,В.~Ломоносова>> \\[.5\baselineskip]
        Юридический адрес: \\
        119991, г.~Москва, ГСП-1, Ленинские горы, 1 \\
        ИНН 7729082090 / КПП 772945026 \\[.5\baselineskip]
        Контактное лицо: \\
        ФИО: Касаткин Сергей Евгеньевич \\
        E-mail: \texttt{sergey.k.mm@gmail.com} \\
        Тел.: +7~(945)~930-77-56 \\[.5\baselineskip]
        \textbf{Декан механико\,--\,математического факультета} \\[5pt]
        \sign{\deanShort} \\[40pt]
        м.~п.
    \end{minipage}
}

\adresstable


% Приложения

\def\appendix#1#2#3{
    \newpage

    \begin{flushright}
        Приложение №#1 \\
        к договору на прохождение практики студентами \\
        Московского государственного университета имени М.\,В.~Ломоносова \\
        в организации (учреждении, предприятии) \\
        от \trim{\date}
    \end{flushright}

    \vskip \baselineskip

    \SECTION{#2}

    #3

    \vskip \baselineskip

    \adresstable
}

\appendix{1}{Список обучающихся, направляемых в профильную организацию}{
    \newcounter{studentno}
    \def\addStudent#1#2#3{
        \stepcounter{studentno}
        \thestudentno &
        #1 \linebreak #2 \linebreak #3 &
        Механико\,--\linebreak{}ма\-те\-ма\-ти\-чес\-кий факультет &
        01.05.01. Фундаментальные математика и механика &
        \dateRangeCell{\practiceBeginDate}{\practiceEndDate} \\
        \hline
    }
    \begin{tabularx}{\textwidth}{|p{1.5em}|X|p{7em}|p{10em}|p{8em}|}
        \hline
            \textbf{№ \newline п/п} &
            \textbf{ФИО обучающегося} &
            \textbf{Факультет, \newline Институт, \newline Филиал, курс} &
            \textbf{Направление \newline подготовки \newline (специальность)} &
            \textbf{Сроки \newline практической \newline подготовки} \\
        \hline
        \studentsList
    \end{tabularx}
}

\appendix{2}{Помещения профильной организации}{
    \newcounter{roomno}
    \def\addRoom#1#2{
        \stepcounter{roomno}
        \theroomno & #1 & #2 \\
        \hline
    }
    \begin{tabularx}{\textwidth}{|p{1.5em}|X|X|}
        \hline
            \textbf{№ \newline п/п} &
            \textbf{Список помещений, оборудования \newline и технических средств} &
            \textbf{Адрес местонахождения помещения} \\
        \hline
        \roomsList
        \ifnum\value{roomno}=0
            & & \\ \hline
        \fi
    \end{tabularx}
}

\end{document}
